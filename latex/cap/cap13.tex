\section{Cap 13 - Problemas de potencial}

Primeira forma de se escrever o divergente do gradiente,
%
\begin{equation}
	\begin{split}
	&\nabla \bullet \left(k \nabla \phi\right) = \left(\mdp{}{x} \quad \mdp{}{y}\right) \bullet \left(k \mdp{\phi}{x} \quad k \mdp{\phi}{y}\right)\\
	&\nabla \bullet \left(k \nabla \phi\right) = \mdp{}{x} \left(k \mdp{\phi}{x}\right) + \mdp{}{y} \left(k \mdp{\phi}{y}\right)
	\end{split}
\end{equation}
%
Segunda forma de se escrever o divergente do gradiente,
%
\begin{equation}
	\begin{split}
		&\nabla \bullet \left(k \nabla \phi\right)
		=
		\begin{bmatrix}
			\mdp{}{x} & \mdp{}{y}
		\end{bmatrix}
		\begin{bmatrix}
		\mdp{\phi}{x}\\
		\mdp{\phi}{y}
		\end{bmatrix}\\
		&\nabla \bullet \left(k \nabla \phi\right) = \mdp{}{x} \left(k \mdp{\phi}{x}\right) + \mdp{}{y} \left(k \mdp{\phi}{y}\right)
	\end{split}
\end{equation}