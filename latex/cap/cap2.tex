\section{Formulação variacional}
%
Para considerando a equação diferencial:
%
\begin{equation}
	 \frac{d^2u}{dx^2} + f(x) = 0
	 \label{eq1:eq_dif}
\end{equation}
%
A formulação variacional pode-se ser obtida multiplicado e integrando a eq. (\ref{eq1:eq_dif}) em todo o domínio $L$.
%
\begin{equation}
	\int_0^L\left( \frac{d^2u}{dx^2} + f(x) \right) w(x) dx = 0
\end{equation}
%
Usando a regra da soma de integrais temos,
%
\begin{equation}
	\int_0^L \frac{d^2u}{dx^2} w(x) dx + \int_0^L f(x) w(x) dx= 0
\end{equation}
%
Focando apenas no primeiro termo,
%
\begin{equation}
	\int_0^L \frac{d^2u}{dx^2} w(x) dx  = \frac{du}{dx} w(x) \mid_0^L - \int_0^L \frac{du}{dx} \frac{dw}{dx} dx 
	\label{eq1:integral_por_partes}
\end{equation}
%
O que leva à,

\begin{equation}
	\frac{du}{dx} w \mid_0^L - \int_0^L \frac{du}{dx} \frac{dw}{dx} dx  + \int_0^L f(x) w(x) dx= 0
\end{equation}
%
\subsection{Integral por partes}
%
Para chegar na eq. (\ref{eq1:integral_por_partes}) considere primeiro a regra do produto parada derivas para duas funções genéricas $a(x)$ e $b(x)$:
%
\begin{equation}
	\frac{d(ab)}{dx} = \frac{da}{dx} b + \frac{db}{dx} a
\end{equation}
%
Integrando em ambos os lados,
%
\begin{equation}
	\int_0^L \frac{d(ab)}{dx} dx = \int_0^L \frac{da}{dx} b dx + \int_0^L \frac{db}{dx} a dx
\end{equation}
%
A integral e a deriva se cancelam no primeiro termo,
%
\begin{equation}
	ab|_0^L = \int_0^L \frac{da}{dx} b dx + \int_0^L \frac{db}{dx} a dx
\end{equation}
%
Considerando agora que $a(x) = \frac{du}{dx}$ e $b(x) = w(x)$, portanto 
%
\begin{equation}
	\frac{du}{dx} w|_0^L = \int_0^L \frac{d^2u}{dx^2} w dx + \int_0^L \frac{dw}{dx} \frac{du}{dx} dx
	\label{eq1:formu_var}
\end{equation}
%
levando assim  a eq (\ref{eq1:integral_por_partes}).
%
\subsection{Resíduos ponderados}
%
A aproximando $u(x)$ por $\hat u(x)$ eq. (\ref{eq1:eq_dif}) não e mais resolvida de maneira exata sobrenado um resíduo $R(x)$,
%
\begin{equation}
	\frac{d^2 \hat u}{dx^2} + f(x) = R(x)
\end{equation}
%
A ideia é reduzir este resíduo usando uma aproximação da função peso $\hat w(x)$.
%
\begin{equation}
	\int_0^L R(x) \hat w(x) dx=  \int_0^L\left( \frac{d^2\hat u}{dx^2} + f(x) \right) \hat w(x) dx = 0
\end{equation}
%
Os métodos do Elementos Finitos, Diferença Finita e Volume Finitos são gerados pelas diferentes escolhas das funções $w$. Usando a formulação variacional eq. \ref{eq1:formu_var} temos,
%
\begin{equation}
	\int_0^L \frac{d \hat u}{dx} \frac{d \hat w}{dx} dx  =  	\frac{d \hat u}{dx} \hat w \mid_0^L + \int_0^L f(x) \hat w(x) dx
	\label{eq1:Residuo_ponderados}
\end{equation}
%
\subsection{Aproximações}
%
As aproximações de $\hat u(x)$ e $\hat w(x)$ são dada por,
%
\begin{align}
	&\hat u(x) = \sum_{j = 1}^n N_j(x) u_j\\
	&\hat w(x) = \sum_{i = 1}^n N_i(x) w_i
\end{align}
%
onde $n$ é o número de pontos na malha. Substituindo $\hat w(x)$ na eq. (\ref{eq1:Residuo_ponderados})
%
\begin{equation}
\begin{split}
	\int_0^L \frac{d \hat u}{dx} \frac{d}{dx} \left(\sum_{i = 1}^n N_i(x) w_i\right) dx = \frac{d \hat u}{dx} \left(\sum_{i = 1}^n N_i(x) w_i\right) \mid_0^L  \\ + \int_0^L f(x)\left(\sum_{i = 1}^n N_i(x) w_i\right) dx
\end{split}
\end{equation}
%
Para o primeiro termo temos,
%
\begin{equation}
\begin{split}
	\int_0^L \frac{d \hat u}{dx} \frac{d}{dx} \left(\sum_{i = 1}^n N_i(x) w_i\right) dx =	\int_0^L \frac{d \hat u}{dx}  \frac{d}{dx} \left( N_1 w_1 + ... + N_n w_n \right) dx\\
	=\int_0^L \frac{d \hat u}{dx}  \left( \frac{dN_1}{dx} w_1 + ... + \frac{dN_n}{dx} w_n \right) dx\\
	=\int_0^L \frac{d \hat u}{dx}  \frac{dN_1}{dx} w_1 dx + ... + \int_0^L \frac{d \hat u}{dx}  \frac{dN_n}{dx} w_n dx\\
	=w_1 \int_0^L \frac{d \hat u}{dx}  \frac{dN_1}{dx} dx + ... + w_n \int_0^L \frac{d \hat u}{dx}  \frac{dN_n}{dx}  dx
\end{split}
\label{eq1:p_termo}
\end{equation}
%
Para o segundo termo temos,
%
\begin{equation}
	\begin{split}
	\frac{d \hat u}{dx} \left(\sum_{i = 1}^n N_i(x) w_i\right) \mid_0^L = w_1  \frac{d \hat u}{dx} N_1(x) \mid_0^L + ... + w_n \frac{d \hat u}{dx} N_n(x) \mid_0^L
	\end{split}
\label{eq1:s_termo}
\end{equation}
%
Para o terceiro termo temos,
%
\begin{equation}
	\begin{split}
		\int_0^L f(x)\left(\sum_{i = 1}^n N_i(x) w_i\right) dx = \int_0^L f(x)\left(N_1(x) u_1 + ... + N_n(x) w_n\right) dx\\
		= w_1 \int_0^L f(x) N_1(x) dx + ... + w_n \int_0^L f(x) N_n(x) dx
	\end{split}
\label{eq1:t_termo}
\end{equation}
%
%---------------------------------------------------------------------------
%
Juntando as eqs. (\ref{eq1:p_termo}), (\ref{eq1:s_termo}) e (\ref{eq1:t_termo}) temos,
%
\begin{equation}
	\begin{split}
    w_1 \int_0^L \frac{d \hat u}{dx}  \frac{dN_1}{dx} dx + ... + w_n \int_0^L \frac{d \hat u}{dx}  \frac{dN_n}{dx}  dx =\\
    w_1  \frac{d \hat u}{dx} N_1(x) \mid_0^L + ... + w_n \frac{d \hat u}{dx} N_n(x) \mid_0^L \\
    + w_1 \int_0^L f(x) N_1(x) dx + ... + w_n \int_0^L f(x) N_n(x) dx
	\end{split}
\end{equation}
%
Isolando agora os $w_i$,

\begin{equation}
	\begin{split}
		&w_1 \left(\int_0^L \frac{d \hat u}{dx}  \frac{dN_1}{dx} dx -  \frac{d \hat u}{dx} N_1(x) \mid_0^L -\int_0^L f(x) N_1(x) dx \right)\\
		&+ ... + \\
		&w_n \left(\int_0^L \frac{d \hat u}{dx}  \frac{dN_n}{dx} dx -  \frac{d \hat u}{dx} N_n(x) \mid_0^L -\int_0^L f(x) N_n(x) dx \right) = 0
	\end{split}
	\label{eq1:somatorio_i}
\end{equation}
%
Para que a eq (\ref{eq1:somatorio_i}) seja sempre verdade ou os $w_i$ são todos nulos ou as expressões são nulas. Fazer todos $w_i$ nulos não tem utilidade, logo a melhor opção é fazer todas as expressões dentro dos parenteses nulas. Assim nos temos um conjunto $n$ equações,
%
\begin{equation}
	\begin{split}
		&\int_0^L \frac{d \hat u}{dx}  \frac{dN_1}{dx} dx -  \frac{d \hat u}{dx} N_1(x) \mid_0^L -\int_0^L f(x) N_1(x) dx  = 0\\
		&...\\
		&\int_0^L \frac{d \hat u}{dx}  \frac{dN_n}{dx} dx -  \frac{d \hat u}{dx} N_n(x) \mid_0^L -\int_0^L f(x) N_n(x) dx  = 0\\
	\end{split}
	\label{eq1:equacoes_i}
\end{equation} 
%
Ainda falta substituir o $\hat u$, para a primeira integral,
%
\begin{equation}
	\begin{split}
	\int_0^L \frac{d \hat u}{dx}  \frac{dN_i}{dx} dx = \int_0^L \frac{d}{dx}\left(\sum_{j = 1}^n N_j(x) u_j\right)  \frac{dN_i}{dx} dx\\
	 = \int_0^L \frac{d}{dx} \left( N_1(x) u_1 + ... N_n(x) u_n \right) \frac{dN_i}{dx} dx\\
	 = \int_0^L \frac{d}{dx} \left( N_1(x) u_1 \right) \frac{dN_i}{dx} dx + ... + \int_0^L \frac{d}{dx} \left( N_n(x) u_n \right) \frac{dN_i}{dx} dx\\
	 = u_1 \int_0^L \frac{N_1}{dx} \frac{dN_i}{dx} dx + ... + u_n \int_0^L \frac{d N_n}{dx} \frac{dN_i}{dx} dx
	\end{split}
\end{equation} 
%
Substituindo nas eqs. (\ref{eq1:equacoes_i}) chegamos ao sistema final de equações.
%
\begin{equation*}
	\begin{split}
		& u_1 \int_0^L \frac{N_1}{dx} \frac{dN_i}{dx} dx + ... + u_n \int_0^L \frac{d N_n}{dx} \frac{dN_i}{dx} dx -  \frac{d \hat u}{dx} N_1(x) \mid_0^L -\int_0^L f(x) N_1(x) dx  = 0\\
		&...\\
		& u_1 \int_0^L \frac{N_1}{dx} \frac{dN_n}{dx} dx + ... + u_n \int_0^L \frac{d N_n}{dx} \frac{dN_n}{dx} dx -  \frac{d \hat u}{dx} N_n(x) \mid_0^L -\int_0^L f(x) N_n(x) dx  = 0\\
	\end{split}
\end{equation*}
%
Definido os $k_{ij}$ e $f_{i}$, 
\begin{equation}
	\begin{split}
	&k_{ij} = \int_0^L \frac{N_j}{dx} \frac{dN_i}{dx}\\
	&f_{i} = \int_0^L f(x) N_i(x) dx + \frac{d \hat u}{dx} N_i(x) \mid_0^L
	\end{split}
\end{equation}
%
Pode-se escrever os sistema de equações final como,

\begin{equation}
	\begin{split}
		& u_1 k_{11} + ... + u_n k_{1n} -  f_i  = 0\\
		&...\\
		& u_1 k_{n1} + ... + u_n k_{nn} -  f_n  = 0
	\end{split}
\end{equation}
%---------------------------------------------------------------------------
