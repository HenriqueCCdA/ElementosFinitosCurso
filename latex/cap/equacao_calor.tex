%
\section{Equação de transferência de calor }
%
A equação de calor é dada por,
%
\begin{equation}
	\frac{\partial( \rho c_p T )}{\partial t} = \bar \nabla \bullet ( k \bar \nabla T ) + Q(x, y, z, t)
\end{equation}
%
onde $\rho$ é massa especifica, $c_p$ é o calor especifico, $k$ é coeficiente de condutividade térmica e $Q(x, y, z, t)$ é um termo fonte de geração de calor.  
%
Considerando o regime permanente temos,
%
\begin{equation}
	\bar \nabla \bullet ( k \bar \nabla T ) + Q(x, y, z) = 0
\end{equation}
%
Agora considerando que o $k$ é constante,
%
\begin{equation}
	\begin{split}
		k\left(\bar \nabla \bullet \bar \nabla T\right) + Q(x, y, z) = 0\\	 
		k \nabla^2 T + Q(x, y, z) = 0
	\end{split}
\end{equation}
%
Dividindo $k$
%
\begin{equation}
	\begin{split}
		\nabla^2 T = -Q(x, y, z)/ k = 0\\	 
		\nabla^2 T = Q^*(x, y, z)
	\end{split}
\end{equation}
%
A equação abaixo é conhecida com Equação de Poisson
%
\begin{equation}
	\nabla^2 T = \frac{\partial^2 T}{\partial x^2} + \frac{\partial^2 T}{\partial y^2} + \frac{\partial^2 T}{\partial z^2} = Q^*(x, y, z)	
\end{equation}
%
A equação de Laplace é a equação de Poisson com termo fonte nulo,
%
\begin{equation}
	\nabla^2 T = \frac{\partial^2 T}{\partial x^2} + \frac{\partial^2 T}{\partial y^2} + \frac{\partial^2 T}{\partial z^2} = 0
\end{equation}
%
Para um problema 1D a equações de Poisson fica,
%
\begin{equation}
	\frac{\partial^2 T}{\partial x^2} = \frac{d^2 T}{d x^2} = Q^*(x)	
\end{equation}
%
%
\section{Equação de transporte de calor }
%
A equação de transporte de calor é dada por,
%
\begin{equation}
	\frac{\partial( \rho c_p T )}{\partial t} + \bar \nabla \bullet (\rho c_p \pmb{\bar v} T ) = \bar \nabla \bullet ( k \bar \nabla T ) + Q(x, y, z, t)
\end{equation}