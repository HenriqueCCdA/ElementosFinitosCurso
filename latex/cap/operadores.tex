\section{Operadores Diferencias}
%
\subsection{Gradiente}
%
O operado diferencial gradiente é definido por,
%
\begin{equation}
	grad() = \bar \nabla() = \frac{\partial()}{\partial x} \pmb{\hat i} + \frac{\partial()}{\partial y} \pmb{\hat j} + \frac{\partial()}{\partial z} \pmb{\hat z} = \left[\frac{\partial()}{\partial x}\; \frac{\partial()}{\partial y}\; \frac{\partial()}{\partial z}  \right]
\end{equation}
%
O gradiente de um escalar é um vetor, por exemplo, o gradiente de $u$ é um vetor dados por,
%
\begin{equation}
	\bar \nabla(u) = \frac{\partial u}{\partial x} \pmb{\hat i} + \frac{\partial u}{\partial y} \pmb{\hat j} + \frac{\partial u }{\partial z} \pmb{\hat z}
\end{equation}
%
Uma aplicação prática do gradiente é fluxo de calor que é dado pelo gradiente da Temperatura:
%
\begin{equation}
	\bar q = - k \bar \nabla(T) = - k \left(\frac{\partial T}{\partial x} \pmb{\hat i} + \frac{\partial T}{\partial y} \pmb{\hat j} + \frac{\partial T}{\partial T} \pmb{\hat z}\right)
\end{equation}
%
Outra forma de escrever o gradiente é usando notação indicial
%
\begin{equation}
	\bar \nabla(u) = \frac{\partial u}{\partial x_1} \pmb{\hat i_1} + \frac{\partial u}{\partial x_2} \pmb{\hat i_2} + \frac{\partial u}{\partial x_3} \pmb{\hat i_3} = \sum_{j = 1}^3 \frac{\partial u}{\partial x_j} \pmb{\hat i_j} =  \frac{\partial u}{\partial x_j} \pmb{\hat i_j} = u_{,j} \pmb{\hat i_j}
\end{equation}
%
\subsection{Divergente}
%
O operado diferencial divergente é definido por,
%
\begin{equation}
	div() = \bar \nabla \bullet () = \frac{\partial()}{\partial x} + \frac{\partial()}{\partial y} + \frac{\partial()}{\partial z} 
\end{equation}
%
Por exemplo, considere o vetor velocidade $\pmb{\bar v}$ dado por,
%
\begin{equation}
	\pmb{\bar v} =	u \pmb{\hat i} + v \pmb{\hat j} + w \pmb{\hat z}
\end{equation}
%
O divergente é dado,
%
\begin{equation}
	\bar \nabla \bullet \pmb{\bar v} = \frac{\partial u}{\partial x} + \frac{\partial v}{\partial y} + \frac{\partial w}{\partial z} 
\end{equation}
%
Em notação indicial a velocidade fica,
%
\begin{equation}
	\pmb{\bar v} =	u_1 \pmb{\hat i_1} + u_2 \pmb{\hat i_2} + u_3 \pmb{\hat i_3}
\end{equation}
%
E o divergente
%
\begin{equation}
	\bar \nabla \bullet \pmb{\bar v} = \frac{\partial u_1}{\partial x_1} + \frac{\partial u_2}{\partial x_2} + \frac{\partial u_3}{\partial x_3} = \frac{\partial u_j}{\partial x_j} = u_{j,j}
\end{equation}
%
\subsection{Operado laplaciano}
%
O operado laplaciano é o divergente do gradiente,
\begin{equation}
	div(grad()) = \bar \nabla \bullet \bar \nabla() = \frac{\partial^2()}{\partial x^2} + \frac{\partial^2()}{\partial y^2} + \frac{\partial^2()}{\partial z^2} = \nabla^2 ()
\end{equation}




