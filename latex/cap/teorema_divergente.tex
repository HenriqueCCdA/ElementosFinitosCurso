\section{Teorema do divergente}

A teorema do divergente é dado por:
%
\begin{equation}
	\int_{\Omega} \nabla \bullet \pmb{\bar F} d \Omega = \int_{\Gamma} \pmb{\bar F} \bullet \pmb{\bar n} d \Gamma
\end{equation} 
%
O vetor $\pmb{\bar F}$ e  $\pmb{\bar n}$ podem ser escritos como,
%
\begin{equation}
	\begin{split}
	&\pmb{\bar F} = \left(F_x, F_y, F_z\right)\\
	&\pmb{\bar n} = \left(n_x, n_y, n_z\right)
	\end{split}
\end{equation} 
%
Assim da para escrever o teorema do divergente como,
%
\begin{equation}
\int_{\Omega} \left(\mdp{F_x}{x} + \mdp{F_y}{y} + \mdp{F_z}{z}\right) d \Omega = \int_{\Gamma} \left(F_x n_x + F_y n_y + F_z n_z\right) d \Gamma
\end{equation}
%
Pode-se provar ainda que,
%
\begin{equation}
	\begin{split}
		&\int_{\Omega} \mdp{F_x}{x} d \Omega = \int_{\Gamma} F_x n_x d \Gamma\\
		&\int_{\Omega} \mdp{F_y}{y} d \Omega = \int_{\Gamma} F_y n_y d \Gamma\\
		&\int_{\Omega} \mdp{F_z}{z} d \Omega = \int_{\Gamma} F_z n_z d \Gamma
	\end{split}
\end{equation} 
