\section{Equação do Equilíbrio estático}

Equação de equilíbrio é dado por,
%
\begin{equation}
\nabla \bullet \pmb{T} + \pmb{b} = \pmb{0}	
\end{equation}
%
\begin{equation}
	\nabla \bullet \begin{bmatrix}
		\sigma_x	& \tau_{xy}\\ 
		\tau_{xy}   & \sigma_y
	\end{bmatrix}
    +
    \begin{bmatrix}
    b_x	\\
    b_y 
    \end{bmatrix}
     = 
    \begin{bmatrix}
     	0	\\
     	0 
    \end{bmatrix}
\end{equation}
%
\begin{equation}
	\begin{bmatrix}
		\frac{\partial \sigma_x}{\partial x}	+ \frac{\partial \tau_{xy}}{\partial y}\\ 
		\frac{\partial \tau_{xy}}{\partial x} + \frac{\partial \sigma_y}{\partial y}
	\end{bmatrix}
	+
	\begin{bmatrix}
		b_x	\\
		b_y 
	\end{bmatrix}
	= 
	\begin{bmatrix}
		0	\\
		0 
	\end{bmatrix}
\end{equation}
%
Portanto temos duas equações,
%
\begin{equation}
	\begin{split}
	\frac{\partial \sigma_x}{\partial x}	+ \frac{\partial \tau_{xy}}{\partial y} + b_x = 0\\
    \frac{\partial \tau_{xy}}{\partial x} + \frac{\partial \sigma_y}{\partial y} + b_y = 0
	\end{split}
	\label{eq3:equilibrio}
\end{equation}
%

As incógnitas são $\sigma_x$, $\sigma_y$ e $\tau_{xy}$ porém só temos 2 equações. Para solução do problema precisamos que o número de equações seja igual ao número de incógnitas. Para tal existe o método dos deslocamento,neste método  as equações de equilíbrio são escritas em termos do deslocamentos. Para tal vamos relacionar as tensões com as deformações e depois as deformações com os deslocamentos.

Equação constitutiva
%
\begin{equation}
\begin{bmatrix}
	\sigma_x\\ 
	\sigma_y\\
	\tau_{xy}
\end{bmatrix}
=
\begin{bmatrix}
	a & b& 0\\ 
	b & a& 0\\ 
	0 & 0& c\\ 
\end{bmatrix}
\begin{bmatrix}
    \epsilon_x\\ 
	\epsilon_y\\ 
	\gamma_{xy}\\  
\end{bmatrix}
\end{equation}
%
Multiplicando 
%
\begin{equation}
	\begin{split}
		&\sigma_x = a \epsilon_x + b \epsilon_y\\
		&\sigma_y = b \epsilon_x + a \epsilon_y\\
		&\tau_{xy} = c \gamma_{xy}
	\end{split}
\end{equation}
%
Relação da deformação com os deslocamentos,
%
\begin{equation}
	\begin{split}
		&\epsilon_x = \mdp{u}{x}\\
		&\epsilon_y = \mdp{v}{y}\\
		&\gamma_{xy} = \mdp{u}{y} + \mdp{v}{x}
	\end{split}
\end{equation}
%
Agora as tensões podem ser escritas em função do deslocamentos
\begin{equation}
	\begin{split}
		&\sigma_x = a \mdp{u}{x} + b \mdp{v}{y}\\
		&\sigma_y = b \mdp{u}{x} + a \mdp{v}{y}\\
		&\tau_{xy} = c \left(\mdp{u}{y} + \mdp{v}{x}\right)
	\end{split}
\end{equation}
%
Calculando as derivadas
%
\begin{equation}
	\begin{split}
		&\mdp{\sigma_x}{x} = a \mdpv{u}{x} + b \mdph{v}{x}{y} \\
		&\mdp{\sigma_y}{y} = b \mdph{u}{x}{y} + a \mdpv{v}{y}\\
		&\mdp{\tau_{xy}}{x} = c \left(\mdph{u}{x}{y} + \mdpv{v}{x}\right)\\
		&\mdp{\tau_{xy}}{y} = c \left(\mdpv{u}{y} + \mdph{v}{x}{y}\right)
	\end{split}
\end{equation}
%
Substituindo na equação (\ref{eq3:equilibrio})
%
\begin{equation}
	\begin{split}
		a \mdpv{u}{x} + b \mdph{v}{x}{y} + c \mdpv{u}{y} + c \mdph{v}{x}{y} + b_x = 0\\
		c \mdph{u}{x}{y} + c \mdpv{v}{x} +  b \mdph{u}{x}{y} + a \mdpv{v}{y}  + b_y = 0		
	\end{split}
\end{equation}
%
Rearranjo do as equações,
%
\begin{equation}
	\begin{split}
		a \mdpv{u}{x} + c \mdpv{u}{y} + (b + c) \mdph{v}{x}{y} + b_x = 0\\
		c \mdpv{v}{x} + a \mdpv{v}{y} + (b + c) \mdph{u}{x}{y} + b_y = 0		
	\end{split}
\end{equation}
%
Agora temos 2 equações e 2 incógnitas ($u$, $v$) , problema esta matematicamente fechado. 

\subsection{Condição de contorno natural}

Condição de contorno natural é quando prescrevemos força no contorno.
%
\begin{equation}
	\begin{bmatrix}
		\bar t_x\\ 
		\bar t_y
	\end{bmatrix}
	=
	\begin{bmatrix}
		\sigma_x & \tau_{xy}\\ 
		\tau_{xy}& \sigma_y 
	\end{bmatrix}
	\begin{bmatrix}
		n_x\\ 
		n_y\\  
	\end{bmatrix}
\end{equation}
%
Multiplicando
%
\begin{equation}
	\begin{bmatrix}
		\bar t_x\\ 
		\bar t_y
	\end{bmatrix}
	=
	\begin{bmatrix}
		\sigma_x n_x + \tau_{xy} n_y\\ 
		\tau_{xy} n_x+ \sigma_y n_y\\  
	\end{bmatrix}
\end{equation}
%
Escrevendo em forma de equações
%
\begin{equation}
	\begin{split}
		\bar t_x =  \left( a \mdp{u}{x} + b \mdp{v}{y} \right) n_x + c \left(\mdp{u}{y} + \mdp{v}{x}\right) n_y\\
		\bar t_y = c \left(\mdp{u}{y} + \mdp{v}{x}\right) n_x + \left( b \mdp{u}{x} + a \mdp{v}{y} \right) n_y
	\end{split}
\end{equation}
%
\subsection{Matriz de coeficientes}
%
Calculo de $\mathbf{B_i}$,
%
\begin{equation}
\begin{split}
&\mathbf{B_i} = \pounds  \mathbf{N_i}	\\
&\mathbf{B_i} = \begin{bmatrix}
	\mdp{}{x} & 0 \\ 
	0         & \mdp{}{y} \\
	\mdp{}{y} & \mdp{}{x}       
\end{bmatrix}
\begin{bmatrix}
	N_i & 0\\ 
	0   & N_i
\end{bmatrix}\\
&\mathbf{B_i} = \begin{bmatrix}
	\mdp{N_i}{x} & 0 \\ 
	0         & \mdp{N_i}{y} \\
	\mdp{N_i}{y} & \mdp{N_i}{x}       
\end{bmatrix}
\end{split}
\end{equation}
%
Calculo de $\mathbf{B_i} \mathbf{D} \mathbf{B_j}$,
%
\begin{equation}
	\begin{split}
&\mathbf{B_i} \mathbf{D} \mathbf{B_j}
	=
	\begin{bmatrix}
		\mdp{N_i}{x} & 0             & \mdp{N_i}{y}\\ 
		0            & \mdp{N_i}{y}  & \mdp{N_i}{x}
	\end{bmatrix}
	\begin{bmatrix}
		a & b& 0\\ 
		b & a& 0\\ 
		0 & 0& c\\ 
	\end{bmatrix}
	\begin{bmatrix}
	\mdp{N_j}{x} & 0\\ 
	0            & \mdp{N_j}{y}\\
 	\mdp{N_j}{y} & \mdp{N_j}{x}
	\end{bmatrix}\\
&\mathbf{B_i} \mathbf{D} \mathbf{B_j}
	=
	\begin{bmatrix}
		a \mdp{N_i}{x} & 	b \mdp{N_i}{x} & c \mdp{N_i}{y}\\ 
		b \mdp{N_i}{y} & 	a \mdp{N_i}{y} & c \mdp{N_i}{x}
	\end{bmatrix}
	\begin{bmatrix}
		\mdp{N_j}{x} & 0\\ 
		0            & \mdp{N_j}{y}\\
		\mdp{N_j}{y} & \mdp{N_j}{x}
	\end{bmatrix}\\
&\mathbf{B_i} \mathbf{D} \mathbf{B_j}
=
\begin{bmatrix}
	a \mdp{N_i}{x}\mdp{N_j}{x} + c \mdp{N_i}{y}\mdp{N_j}{y} & 	b \mdp{N_i}{x}\mdp{N_i}{y} + c \mdp{N_i}{y}\mdp{N_j}{y}\\ 
	b \mdp{N_i}{y}\mdp{N_j}{x} + c \mdp{N_i}{x}\mdp{N_j}{y} & 	a \mdp{N_i}{y}\mdp{N_j}{y} + c \mdp{N_i}{x}\mdp{N_j}{x}
\end{bmatrix}
\end{split}
\end{equation}